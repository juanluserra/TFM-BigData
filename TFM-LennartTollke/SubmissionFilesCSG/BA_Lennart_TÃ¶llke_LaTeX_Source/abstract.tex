\chapter*{Abstract Deutsch}
\addcontentsline{toc}{chapter}{Abstract}


%===============================================================================
%%%%%%%%%%%%%%%%%%%%%%%%%%%%%%%%
% Recipe to do an abstract:
%%%%%%%%%%%%%%%%%%%%%%%%%%%%%%%%
%(1) Context
%(2) Research Gap
%(3) Goals, Objectives
%(4) Methodology, how
%(5) Results

% Example:
%Abstract
%(1) In the context of energy efficiency in computer networks, a significant number of solutions ranging from protocols and functionalities to energy efficiency-oriented management applications have been proposed. 
%(2) However, the characteristics of environments to develop and validate such solutions are not as discussed as the solutions themselves. 
%(3) Considering this, this work proposes an emulation environment to develop and validate energy efficiency-oriented solutions, as well as discuss their specific characteristics. 
%(4) Thus, three functionalities of different network scopes are implemented, Adaptive Link Rate (interface level), Syncronized Coalescing (device level) and SustNMS (network level) in the Mininet emulation environment using the implemented software-defined networks paradigm on the POX controller. 
%(5) The environment is validated by comparing the energy savings achieved by these features in a topology inspired by the National Research Network (RNP).

\selectlanguage{german}

Kognitive Kriegsf\"uhrung ist eine aktuell aufkommende Bedrohung, deren Praktiken auf die Beeinflussung der Kognition des Gegners abzielen. Elektromagnetische Angriffe sind eine technologische Variante dieser Kriegsf\"uhrung, welche die gesunde Gehirnaktivit\"at durch elektromagnetische Strahlung st\"ort. Damit in Verbindung gebrachte Zwischenf\"alle haben gezeigt, dass dies zu einer Reihe von Symptomen wie Kopfschmerzen, Schwindelgef\"uhlen und Ged\"achtnisst\"orungen f\"uhren kann.\\
Trotz der Schwere dieser Angriffe und ihres zunehmenden Bedrohungspotenziales, existieren bisher keine Ans\"atze f\"ur Gegenmassnahmen. Diese Arbeit beinhaltet daher den Versuch, auf Grundlage von Gehirn-Computer-Schnittstellen (BCIs) eine L\"osung zu entwickeln, die die Auswirkungen solcher Angriffe abschw\"achen kann. Der Schwerpunkt liegt dabei auf dem kognitiven Prozess der Ged\"achtniskonsolidierung.\\
Durch Sichtung der verf\"ugbaren Literatur werden elektromagnetische Angriffe und die M\"oglichkeiten von BCIs untersucht. Darauf aufbauend folgen spezifischere Untersuchungen, die mit Hilfe von Gehirnsimulationen durchgef\"uhrt werden.\\
Obwohl es im Rahmen dieser Arbeit nicht m\"oglich ist, eine spezifische Gegenmassnahme vorzuschlagen, findet die Untersuchung notwendiger Aspekte f\"ur deren Entwicklung statt. Insbesondere k\"onnen einige potenziell zugrundeliegende Mechanismen elektromagnetischer Angriffe identifiziert und in die Simulation integriert werden. Daraus ergibt sich eine deutliche Abweichung von gesunder Hirnaktivit\"at, die auch mit den berichteten Beeintr\"achtigungen des Ged\"achtnisses in Verbindung gebracht werden kann.


\newpage

\selectlanguage{english}
\vspace*{50pt} % Add some vertical space before the title
{\Huge\bfseries Abstract English} % Match chapter title formatting
\vspace*{40pt} % Space after the title

%(1) Context
Cognitive Warfare is an emerging threat that encompasses practices to interfere with the adversaries' cognition. Electromagnetic attacks are a technological example of this, where with the help of electromagnetic radiation, normal brain activity is disrupted. This can lead to a range of symptoms like headaches, dizziness, and memory impairments, which were reported from a series of related health incidents.\\
%(2) Research Gap
Even though such attacks occur and their potential threat only seems to increase, no countermeasures have been proposed so far.
%(3) Goals, Objectives
This work therefore tries to develop a solution, based on brain-computer interfaces that could mitigate the effects of such attacks. Focusing specifically on the cognitive process of memory consolidation.\\
%(4) Methodology, how
By reviewing the available literature, electromagnetic attacks and the capabilities of brain-computer interfaces are explored. This is then used to perform more specific investigations with the help of brain simulations.\\
%(5) Results
Whilst no specific countermeasure can be proposed in the scope of this work, necessary aspects for its development are explored. In particular, potential underlying mechanisms of electromagnetic attacks are identified and integrated into the simulation. This results in a clear deviation from healthy brain activity, which can also be associated with the reported memory impairments. 






