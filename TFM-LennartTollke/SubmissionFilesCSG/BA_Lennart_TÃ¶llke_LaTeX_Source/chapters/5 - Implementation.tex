\chapter{Implementation} % ~ 10 pages
\label{chap:implementation}
% Documentation of my code implementations
This chapter presents and explains the most important implementations of this work, including only what is relevant to the contribution of this work. Thereby, a connection between the theory of the previous chapter and its technological realization shall be made, while focusing mostly on the high-level functionality of the code.\\
According to the simulation model, all the developed functionalities were written in Python \cite{van1995python}. This includes the synthetic input generation, the output analysis, as well as some code related to the attack integration, which is presented in this order. 


\section{Synthetic Input}
% Mention goal and approach to this section
    % developed in 1. model version
The necessity of a synthetic input generation function arose in the context of the first model version \cite{HippSimModel.1}. It was therefore implemented and later integrated into the corresponding model code.

    %\subsection{Generation Function}
    % General approach
        % recreated input equation from design
        % but tried to replace realistic approach and code seamlessly
            % translated into frequency
        % evaluation of results lead to addition of
            % wave_realization_interval and according function
            % integration of variable noise scaling
        % fixed values for parameters that are fixed in model
            % sampling frequency
        % made rest a parameter
    As discussed in the design chapter, this implementation is based on the works of \textcite{Aussel.2021}, who proposed an equation for synthetic input. However, the implementation was adjusted to fit seamlessly into the existing code. Primarily resulting in the decision to not integrate the input by injecting a current into the stimulated neurons, but rather translate the input equation into frequencies that define the firing rate of those neurons. Since such a translation of a stimulus pattern into firing frequencies of neurons is also performed with the realistic input files, it was possible to adapt some aspects of the approach there and then supply the model with the same type of input parameter. This further avoided the necessity of more fundamental changes to the model, which would have been hard to validate.\\
    The developed function includes the "is\_in\_correct\_cycle" parameter and therefore is an algorithmic representation of the second equation presented in the design. Values like the sampling frequency that are tuned to the model configuration were defined inside the function whilst everything else remained a variable parameter.\\

    \begin{lstlisting}[caption={Generate Input Function}]
    def generate_input(A1, f1, wave_realization_interval, sim_time):
        
        record_dt = 1. / 1024 * second  # Sampling interval
        t0 = 0.250  # Start time of input
    
        # Array of sample times
        times = np.arange(0 * second, sim_time, record_dt)
    
        # Generate square wave
        input_values = np.zeros_like(times)
    
        def is_in_correct_cycle(input_frequency, current_time, wave_realization_int):
            ms_per_cycle = 1 / input_frequency
            wave_number = int((current_time - t0) / ms_per_cycle)
            return wave_number % wave_realization_int == 0
    
        for i, t in enumerate(times):
            if t >= t0 and np.sin(2 * np.pi * f1 * (t - t0)) >= 0 and is_in_correct_cycle(f1, t, wave_realization_interval):
                input_values[i] = A1
    
    
        # Normalize values to range [0, 1]
        input_normalized = (input_values - min(input_values)) / (max(input_values) - min(input_values))
    
        # Add noise
        input_noisy = (5 / 6) * input_normalized + (1 / 6) * np.random.rand(len(input_normalized))
    
        # Scale to a maximum frequency of 200 Hz
        max_rate = 200 * Hz
        input_scaled = input_noisy * max_rate
    
        input_timed = TimedArray(input_scaled, dt=record_dt)
    
        return input_timed
    \end{lstlisting}
    The parameters of the function allowed for the adjustment and exploration of all important aspects of the input. Whilst \textit{A1} and \textit{f1} represent the same values that can be found in the input equation, the \textit{wave\_realization\_interval} stands for \(w\). Further, the \textit{sim\_time} parameter was necessary to represent the configured simulation runtime in seconds and ensures that the input is generated in the required length. The function returns a brian2 "TimedArray", containing an input frequency at every sample time step. This array could then be used to replace the ones that would otherwise be created from realistic input files. 

    %\subsection{Realistic Input Integration}?


\section{Output Analysis}
% Connect to Design section
% mention i based it on Event_detection_Aussel.py
    % changes i made
% explain structure of section
    % sharp wave detection function
    % event detection function employing it
From a technical standpoint, the output analysis incorporates two different procedures. First, the LFP data has to be processed, such that relevant aspects are revealed. Only then can the data be structured and displayed in a way that facilitates the extraction of relevant information.\\
The approach to the data processing was based on the works \textcite{Aussel.2018}, who supplied a basic event detection function in their first model code \cite{HippSimModel.1}. However, this code was altered significantly, to facilitate the data analysis of this work. For reasons of modularity and encapsulation, the procedure was split into two functions, that relate to the two SWR detection stages, mentioned in the design. The first one identifies sharp waves, whilst the second one analyses their superimposed oscillation frequencies. In the following, their specific implementations are presented.

    \subsection{Sharp Wave Detection}
    % general procedure
    % tell what subfunctions do
    Since the sharp waves are identified with the help of root mean square values of non-overlapping signal windows, these RMS values are calculated first. This was done with a helper function, to enhance readability of the code. Its standard deviation is then derived to specify the threshold condition values. Parts of the signal with RMS values higher than the boundary condition, exceeding the peak threshold at some point, qualify as sharp waves.\\
    
    \begin{lstlisting}[caption={Sharp Wave Detection Function}]
    def sharp_wave_detection(sig, boundary_condition, peak_condition, record_dt):
        # calculation of root-mean-square
        start_plot_time = 50 * msecond
        start_ind = int(start_plot_time / record_dt)
        sig_rms = window_rms(sig[start_ind:] - mean(sig[start_ind:]), int(10 * ms / record_dt))
        sig_std = std(sig_rms)
    
        boundary_value = boundary_condition * sig_std
        peak_value = peak_condition * sig_std
    
        # detection of sharp waves
        begin = 0
        peak = False
        all_sharp_waves = []
    
        for ind in range(len(sig_rms)):
            rms = sig_rms[ind]
            if rms > boundary_value and begin == 0:
                # event start
                begin = ind
            if rms > peak_value and not peak:
                # event fulfills peak condition
                peak = True
            elif rms < boundary_value and peak:
                # sharp wave detected
                sharp_wave_signal = sig[begin + start_ind:ind + start_ind]
                all_sharp_waves.append(sharp_wave_signal)
                begin = 0
                peak = False
            elif rms < boundary_value and not peak:
                # event without sufficient peak
                begin = 0
                peak = False
    
        return all_sharp_waves
    \end{lstlisting}
     % explanation of derived values/function output
     Next to the signal that is analyzed for sharp waves, this function accepts the boundary and peak condition values, defining them. This allows for the convenient evaluation of different values. Since the sampling interval \textit{record\_dt}, needs to align with external values, it was chosen to be a necessary parameter as well.\\
     The return value of the function, labeled \textit{all\_sharp\_waves}, contains lists of the identified sharp wave signals. Their further analysis and categorization is then performed by the following function. 

    \subsection{Event Detection}
    % general procedure, interesting aspects
    Whilst the main contribution of this function is the analysis of event frequencies and their power spectrum, it also serves as the main output processing function. It collects all the relevant data and structures it to facilitate its use by higher-level analysis and representation functions.\\
    In a first step, it retrieves the sharp wave event signals, with the help of the above-explained function \textit{sharp\_wave\_detection} (setting the boundary and peak condition values to \textit{3} respectively \textit{4.5}). Then it performs the frequency analysis of every event, just as specified in the design section. Employing the helper function \textit{frequency\_band\_analysis} in multiple instances.\\
    Given the \textit{low} and \textit{high}-frequency bound parameters, this function applies a band-pass filter to the \textit{event} and derives the power spectrum of the resulting signal. Leading to the output of a \textit{frequencies} and according \textit{power\_spectrum} list, as well as the \textit{filtered\_event} signal.\\
    This function is used at the beginning of the frequency analysis to restrict it to the relevant oscillation range. Then, general data regarding all events is collected, before sharp wave ripples are detected based on the \textit{peak\_frequency} (the frequency that contains the most power). In the end, the power spectrum of every event is derived in the theta, gamma, and ripple bands.\\
    % CODE
    \begin{lstlisting}[caption={Event Detection Function}]
    def event_detection(sig):
        # model specific signal properties
        sample_frequency = 1024 * Hz
        record_dt = 1 / sample_frequency
    
        event_signals = sharp_wave_detection(sig, 3, 4.5, record_dt)
    
        # frequency analysis
        filtered_events = []
        all_spectrum_peaks = []
        all_durations = []
    
        sharp_wave_ripples = []
        sharp_wave_ripple_peaks = []
        sharp_wave_ripple_durations = []
    
        theta_spectrum = []
        gamma_spectrum = []
        ripple_spectrum = []
    
        for event in event_signals:
            # filter in broad frequency range
            frequencies, power_spectrum, filtered_event = frequency_band_analysis(event, 30, 400, sample_frequency)
            if len(frequencies) != 0 and len(power_spectrum) != 0:
                # collect general event data
                filtered_events.append(filtered_event)
                duration = len(event) * record_dt * 1000
                all_durations.append(duration)
                peak_frequency = frequencies[argmax(power_spectrum)]
                all_spectrum_peaks.append(peak_frequency)
    
                # identify sharp wave ripples
                if 100 <= peak_frequency <= 250:
                    sharp_wave_ripples.append(event)
                    sharp_wave_ripple_peaks.append(peak_frequency)
                    sharp_wave_ripple_durations.append(duration)
    
            # collect power of frequency bands
            theta_spectrum.extend(frequency_band_analysis(event, 5, 10, sample_frequency)[1])
            gamma_spectrum.extend(frequency_band_analysis(event, 30, 100, sample_frequency)[1])
            ripple_spectrum.extend(frequency_band_analysis(event, 100, 250, sample_frequency)[1])
    
        # structure result data
        all_event_data = [event_signals, filtered_events, all_spectrum_peaks, all_durations]
        swr_data = [sharp_wave_ripples, sharp_wave_ripple_peaks, sharp_wave_ripple_durations]
        band_spectra = [theta_spectrum, gamma_spectrum, ripple_spectrum]
    
        return all_event_data, swr_data, band_spectra
    \end{lstlisting}
    The sampling frequency was chosen to be statically defined at the top of the function, as it remained the same overall performed simulations. Therefore only the LFP output signal, in the form of a list is required as parameter.\\
    The function output contains all the collected data, sorted and structured as shown in the last code section. Enabling a convenient extraction of relevant aspects for all further analysis and interpretation procedures. 
    
    On the data that is collected with these functions, a proper analysis still needs to be performed, to extract the information of interest. This includes, for the most part, the aggregation and comparison of multiple simulation outputs. But also the representation of important values in a way that facilitates their interpretation. For example with the help of graphs or well-organized text output. Implementations of this are however not directly related to the contribution of this work and are therefore omitted.


% \subsection{Analysis Functions}?
%     \subsubsection{Single Process Analysis}
 %     \subsubsection{Collection Process Analysis}


\section{Electromagnetic Attacks}
% As specified in Desing Chapter
    % Different types of attack implementations
        % most parameter based
        % some (input alterations) Algorithm based
    % Code wise not much to show for parameter based attacks
        % but general parameterization procedure shall be explained
    % Algorithms presented and discussed
As specified in the design chapter, electromagnetic attacks can be realized by adjusting many different simulation aspects. Their technical integration was however facilitated by the newer model version \cite{HippSimModel.2}, as it provides a large parameter space. This allowed all of the attacks to be realized with their help. Leading to very similar and simple implementations, of which only the general procedure and main aspects will be presented.

    %\subsection{Parameter-based Attacks}
    % Explanation of parallel Processing
    % Limited Code presentation
    % Mostly specification of parameter adjustment process and how programmatically realized
    The newer model version can be configured and executed with the help of 3 different files and related approaches. Two of them supply a graphical user interface with different levels of details that may be adjusted. However, they can only be used to initiate a single simulation configuration at a time. Therefore, exclusively the third option, utilizing the \textit{parallel\_processing.py} file, was used for this work.\\
    This approach enabled the scheduling of multiple simulation configurations, which depending on the available processor cores, were also executed in parallel. Thereby significantly increasing the simulation efficiency and amount of data collected. Ultimately allowing for a more robust and reliable evaluation of results. 

    The \textit{parallel\_processing.py} file mainly consists of parameter definitions, incorporating in total 25 different ones. Whilst some are used to define for example the input file location or simulation duration, others are related to neuron or network model aspects. In the following code snipped, all attack-related parameters are shown with their standard, respectively healthy value configuration.\\
    
    \begin{lstlisting}[caption={Example Parameter definition}]
    # Structural parameters
    liste_maxN = [10000] # number of excitatory neurons in the CA1 region (scaling the rest accordingly)
    liste_g_max_e = [60*psiemens] # maximum synaptic conductances of excitatory synapses
    
    # Acetylcholine (sleep/wake) parameters
    liste_gCAN = [(0.5*usiemens*cmeter**-2, 25*usiemens*cmeter**-2)] # (sleep CAN channel conductance, wakefulness CAN channel conductance)
    liste_CAN = ['sleep'] # choose between sleep and wakefulness CAN channel conductances
    liste_G_ACh = [3] # gain applied on some synaptic conductances under cholinergic modulation ('wake' functional connectivity)
    liste_functional_co = ['sleep']  # choose between sleep and wakefulness functional connectivity
    \end{lstlisting}

    As one can observe, every parameter is defined by a list, which allows the assignment of multiple values for it. When the simulation then is executed, the file will create a cross-product of all available parameter values, and schedule a simulation for it. This means every possible combination of list entries, across all the parameters, will be simulated. Also allowing for the definition of the same value multiple times, to collect more data on the same simulation parameters, with a single file configuration.

    
    %\subsection{Algorithm-based Attacks}
    